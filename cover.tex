\documentclass[11pt]{letter}
\usepackage{graphicx,baskervald,microtype}
\usepackage{hyperref,amsmath,xcolor}
\hypersetup{
  colorlinks=true,
  linkcolor=black,
  urlcolor=black
}
\usepackage{pgfplots,marginnote}
\usepackage[top=0.5in,bottom=0.5in,left=1in,right=1in]{geometry}
%\newgeometry{margin=2.5cm}
\newcommand{\omamargin}[1]{\marginnote{\textbf{#1}}[7pt]}

\begin{document}
\reversemarginpar
\pagestyle{empty}
\noindent Dr. XXX\\
\noindent Department of XXX \\
\noindent University of XXX, XXX \\
\noindent email: XXX\\
\noindent +XXX\,XXX\,XXX\,XXX \\


Dear Editor,

Please find enclosed our manuscript titled
{XXX}, which we are submitting for your consideration to be published in \textit{XXX}.

In the manuscript, we answer the question "are polarizable molecular dynamics force fields better for modelling biomembranes (lipid bilayers) than their non-polarizable counterparts and in which ways". This has long thought to be the case as polarizable models have the potential for more accurate description of ionic interactions and the varying dielectric environment across a lipid bilayer. Systematic benchmarking of the models against both high-resolution experimental data and non-polarizable model has, however, been lacking.

We seek to fill this gap by using the NMRlipids open collaboration framework and the related NMRlipids databank. We test the models againts experimental NMR order parameters (describing the conformational space of lipid molecules), small-angle X-ray scattering curves (depicting the fourier tranformation of the electron density across the membrane), NMR relaxation and correlation rates (describing the conformational dynamics), and the change of NMR order parameters when salt is added to the system, which indicate the strength of ion binding. We find that across all these test, the best non-polarizable models beat the polarizable lipid models tested: AMOEBA, CHARMM-Drude2017/CHARMM-Drude2023. Although the polarizable models give good description in some cases, such as the ability of AMOEBA to capture monovalent ion binding and lipid headgroup dynamics, they also have some serious flaws, like the faulty description of the lipid tails in AMOEBA and vastly too slow conformational dynamics of CHARMM-Drude models. It is particularly important to bring attention the last result, since it means one has to run the simulation longer to obtain good quality data.

We believe the test performed in this paper will provide useful reference for anyone looking to use polarizable force field for membrane simulations. We hope it will also encourage the developers of these models to use the experimental data in the parametrization and/or carefully benchmach their models against it. The paper exemplifies the benefits of doing so by pointing out the improvement from the CHARMM-Drude2017 model to the 2023 version.

 In summary, we trust our manuscript fulfills the high standards for publication in %the \textit{Resource} section of
\textit{XXX} and leave it to your expert consideration.\\

%We now leave it for your own consideration.

Sincerely yours,

on behalf of the authors,\\

Dr. XXX

\end{document}